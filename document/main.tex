\documentclass[a4paper,english]{article}

\setlength{\parskip}{\baselineskip}

%% Use utf-8 encoding for foreign characters
\usepackage[T1]{fontenc}
\usepackage[utf8]{inputenc}
\usepackage{babel}

%% Vector based fonts instead of bitmaps
\usepackage{lmodern}

%% Useful
%\usepackage{fullpage} % Smaller margins
\usepackage{enumerate}

\usepackage{graphicx}

%% Theorem
\usepackage{amsthm}

%% Quantum
\usepackage{braket}

%% More math
\usepackage{amsmath}
\usepackage{amssymb}

\graphicspath{  }

%% Document Header
\title{}
\author{}
\date{}

\begin{document}
\maketitle

\section{Biological Background}

Establishing cause and effect in biology is rarely simple. 
Often when we have  found what we think is a necessary and sufficient precondition for a disease phenotype, like a mutation or drug effect, 
it eventually transpires that reality is far less clear cut.
 One major reason for this is the existence of interactions between different modules and pathways within cellular processes. 
In experiments on model organisms, in the genetics of human disease, 
and in the somatic mutations that either promote or inhibit the proliferation of malignant cells, 
we see that the effects of a mutation can be drastically modulated by its context. 


One of the more well formalized forms of interaction is \em{synthetic rescue}.
When the activity of one module or pathway within the cell's mechanisms 
can partially or completely compensate for the complete ablation of another, 
and both need to be inactivated in order to see any phenotype. 





\section{Problem Formulation}



% we construct a simple model of human disease and the PPI

% There are two modules associated with a disease
% each of the modules contains multiple genes
% each patient has a set of mutated genes
% if even one of the genes in a disease module is mutated, that module is considered nonfunctional
% a patient has the disease if and only if both of the modules are nonfunctional

We present a toy model of synthetic rescue and the PPI, in order to illustrate how to .
In this model there is a fixed set of genes which in every individual (henceforth referred to as a 'patient') is mutated or left unchanged at random. 
Each patient's characteristics is completely determined by their set of mutations. 

Some of the genes are collected together in sets called  \em{gene modules}, each of which corresponds to some cellular function or pathway. 
Each module is considered \em{non-functional}, if at least $k$ of the genes in that module are mutated. (In all of the following we will assume $k$=1,
 but we leave room for generalization)  
 Two of these gene modules are selected and are called the disease modules.
  A patient is considered to \em{have the disease} if both of the disease modules are non-functional.%We present a toy model of synthetic rescue and the PPI, in order to illustrate how to .

% this is all happening on a PPI, by the way
% The disease modules are densely connected in the PPI


The problem we are trying to solve essentially has two components %that are challenging on their own

\begin{enumerate}
    \item We do not know the true modules
    \item We would still have a quadratic search problem even if we did know them
\end{enumerate}

$$ $$
% Grover's algorithm let's us look for things
% Grover's algorithm on a network let's us look for ...
% special nodes in regions of special nodes
% double random walks let us look for special pairs of nodes
% in regions of special pairs of nodes
% We just need to specify an interaction hamiltonian
%  that behaves differently on such pairs of nodes


\section{Combinatorics and Linear Algebra}

% we can see that the 
$$ (\bra{i} \otimes \bra{j}) (\ket{D_1} \otimes \ket{D_2}) = 
\begin{cases}
    1,              & \text{if } i \in  D_1 \text{and}  j \in D_2\\
    0,              & \text{otherwise}
\end{cases}
 $$



%
%

%

$$ $$

\section{Disease Modules and The Symmetric Adjacency Matrix Eigenvectors}

% we don't know the exact real modules
% Eigenvectors often capture modules in networks
% This makes sense ebecause of property of the laplacian
% Here's some diagrams showing how it's true visually
% This means that we can build

\section{Construction of the Interaction Hamiltonian}

\subsection{}
%
%
%

$$ \sum_{S_k} \sum_{i,j \in S_k} (\ket{i} \otimes \ket{j})(\bra{i} \otimes \bra{j})  $$
\subsection{}
\subsection{}


\section{Experimental Results}


\end{document}
